\documentclass[12pt]{memoir}
\usepackage{common}

\addbibresource{references}

\title{}
\author{Andrew Head}

\begin{document}

\definition{Title}{Personal, Relevant Background and Future Goals Statement}

\definition{Author}{Andrew Head. PhD Student, Berkeley Institute of Design, UC Berkeley}

I believe that better tools for end user developers can result in better personal software and self-efficacy.
My current research develops tools to improve programmers' abilities to engage with online code.
My past work has demonstrated my commitment to two of the NSF's criteria of Intellectual Merit and Broader Impacts.
\if 0
I have demonstrated Intellectual Merit by writing and Honorable Mention paper for the core conference in end-user software engineering VL/HCC, and publishing my past work in both CHI and ITS, and a record of strong academic performance.
I have demonstrated Broader Impacts through my research focus on end user development, my involvement in the user interface design classroom, and volunteering at a campus maker space.
\fi
My recent work has ready been recognized at a major conference on end-user programming with an Honorable Mention.
As a Head TA for a user interface design course and a volunteer in the campus hackerspace, I help others learn to build meaningful interactive artifacts with software.
In the space below, I describe my contributions in both of these directions chronologically, starting with an overarching statement of purpose outlining the focus of my current research.

\section{Mission Statement and Future Goals}

One of the beliefs underlying much modern HCI research is that people provided with the right digital tools can better serve themselves at lower cost than if someone else were hired to do the task.
Digital tools that people have for building things for themselves can be improved, and can help people adapt the right design and implementation practices that will help them productively and effectively do what they need.
Spreadsheets can enable accountants to perform complex, aggregate calculations (ref); tools for authoring CAD macros can reduce engineers' time replicating features for new models
% My research takes this belief as a starting point.

When I accepted admission to graduate school, I was working as a software engineering intern with a mobile advertising company in Silicon Valley.
At the time, I was motivated to pursue better understand and develop for human needs that could be better supported by novel tools.
I knew that my current skill, awareness of human technical capability and academically-studied tools was insufficient for me to reliably elicit human needs and build upon and contribute to the growing body of theory on design for technical tools.

I already feel closer to that goal, and this remains my core goal for the coming years of graduate school---developing tools and theory hand-in-hand to better support end user developer information seeking and programming.
By the conclusion of my PhD, I hope to have contributed not only to this theory but also concrete, usable development tools that stand as artifacts to fuel research and starting points for other research groups and professional teams to build upon.

\if 0
My research and productive self has been working towards three major ends.
I have made progress towards each of these three ends.
My efforts in the next four years will continue to work towards bringing about success in each of these three ends.
\fi

\section{Intellectual Merit}

Implementing the electronics for interactive device prototypes is time-consuming, error-prone and tedious.
Lamello was part of a sequence of work at Berkeley where we replaced these electronics for input sensing on 3D-printed interactive device bodies with mechanisms that could be sensed by externally attached, non-invasive sensors.
In the case of Lamello, we created 3D-printed mechanisms that made sound when actuated, and an algorithm to sense input events by analyzing the frequencies at which the mechanisms vibrated.
As a co-author for this work, I contributed to refining the algorithm, developing probabilistic and machine learning-based variants on the algorithm, and evaluating its performance.
This work was presented at CHI, one of the premiere conferences in HCI\@.

With inspiration gained from Burnett \& Myers~\cite{burnett_future_2014}, I place an emphasis on grounding the research artifacts I build in established theory.
For example, the micro-explanations we used for Tutorons~\cite{head_tutorons_2015} took inspiration from minimal instruction~\cite{carroll_nurnberg_1990} and layered instruction~\cite{farkas_layering_1998} to build adaptive documentation that inserted additional detail into web tutorials on demand, while retaining only the information that was expected to be immediately relevant and helpful in error recognition and recovery.
This also draws on the principle of attention investment~\cite{blackwell_psychological_2006}---that users may be more likely to engage with and learn from documentation when there is a much smaller investment to be made in locating and understanding the relevant documentation.
\andrew{Probably best to give a quick mention of attention investment and how this project satisfies it.}

This work is currently supported under an NSF CAREER grant on expertise sharing for end user programmers.
The focus of my research and how this ties into the larger theme.
I build tools for end user software engineering.
Currently, I focus on building better documentation, which is heavily used by programmers of all backgrounds, but frankly insufficient for programmers trying to pick up new languages and APIs, or who are trying to approach new types of tasks.
A good deal of this inspirations comes not only from my own development experience as a professional programmer, but also from informal interviews with a handful of friends, and confirmation from literature for both professional programmers and end user programmers (ref) (ref).

In addition to my work on end user software engineering, I have also published as a co-author on fabrication research~\cite{savage_lamello_2015} at CHI one of the premiere conferences in HCI\@.
Fabrication is related to this overarching goal of helping people serve themselves with powerful, expressive digital tools.
I have also published in a conference on intelligent tutoring systems on serious games applied to language learning~\cite{head_tonewars_2014}.

I suspect that an interest in both the disciplines of fabrication research and education will continue to provide an asset into my on-going research.
One source of end-user developers on campus is the Invention lab fabrication space here.
In fact, it is several of these users I spoke with for motivating interviews for the StackSkim interface.
Furthermore, the emphasis of my prior work on integrating best practices observed in the classroom into mobile interfaces is something I would like to carry forward, by applying an understanding of best practices for learning to making interfaces that support long-term cultivation of knowledge for their users.

\section{Broader Impacts}

Of the NSF broader impacts listed in the call for proposals, my current research most closely focuses on
\textbf{increasing public engagement with science and technology}.
\andrew{Continually refer to this theme throughout this topic}.
Locally, I have worked to \textbf{improve STEM education} as a graduate student instructor for an upper division user interface course, a volunteer in the campus hackerspace, and a mentor for sustainability labs at a local high school.

I served as the Head TA for CS160, Berkeley's upper division user interface design course at Berkeley.
CS160 plays a critical role in rounding out an undergraduate's education.
This may be the one rigorous software engineering course students take at Berkeley.
While most courses undergraduates EECS majors take at Berkeley focus on the programming core of algorithms, machine learning, database implementation, and largely technical considerations, user interface design provides several unique challenges to students.
They must brainstorm, pitch, develop and present their own idea.
Through this process, they not only practice need finding, but also problem definition, rapidly learn how to use new APIs for new hardware (in this case the Moto 360 Android smartwatch), and how to effectively convey their work at public presentations.

For this course, I developed new materials to support active learning in a class that emphasizes learning by doing.
I and my co-TA developed new lab-style learning activities with introductions to Android APIs for databases and location services, and tutorials on wireless connectivity between smartwatches and smartphones.
Our sections focused on giving chances for active practice of producing skits and wordplay for presentations, and working through problems related to models of human performance (Fitts' Law).
As a result, we saw historically quite high attendance rates throughout the semester, with around 50\% attendance through 2/3 of the summer term, huge improvements in the quality of student presentations, and some exceptional final projects from students (link to Valkyrie's site from here).

My past work experience has influenced both how I implement my work, and my metrics for success on my work.
I began grad school with an aggregate two years of internships, including two longer-term internships. 
These influence my aims for developing successful projects.
My past industrial experience has helped me to gain the development experience to work on publicly usable software projects.
During a past one-year internship, I gained familiarity with using the UNIX shell and supporting customers on low-level Linux compilation and commands such as kernel debugging and building custom Linux kernels.
During another internship with Quettra, Inc., I developed infrastructure and tools for maintaining, monitoring, scaling and deploying software to thirty servers that were accepting daily traffic from over one million mobile phones.
Through past hobby work, I've written code as part of creative projects that have attracted wide audiences.
With a team of friends, I led programming efforts for \emph{King's Ascent}, a Flash game that gained over 70,000 plays on major game portals.
I'm interested in building shareable artifacts as the result of my research.
This motivated my choice to build Tutorons as backend servers that could be flexibly implemented in any language with a loose architecture, and the implementation of Tutorons viewing as both a JavaScript library that could be included on any arbitrary webpage or superimposed as a browser plugin.
Tutorons is available for download through open source.

In two other roles, I have worked to help people understand new scientific knowledge or enable others to develop more.
These two roles also have two distinctly different flavors of students in terms of their backgrounds and technical proficiency, which highlights the importance of being able to develop conceptual knowledge equally consumable by students and practitioners of many different backgrounds.

As Inventioneer in the Invention Lab at Berkeley, the greatest impact of what I do is allowing others to work on their projects by being present in the lab during hours where no one else is looking after the lab.
On request, I help students brainstorm about electrical components they can use to sense or actuate for their projects, or troubleshoot the lab equipment (3D printers and vinyl cutters).
I regularly hold lab hours during evenings or weekends to enable those developing under time constraints to have more time making.

By co-coordinating the Berkeley Institute of Design lunch, I both introduce visiting professors and PhD students to the research at Berkeley, and also connect Institute members to these scholars.

My recent teaching and outreach focus has focused on sharing an understanding of the work practices and technical tools necessary to build meaningful technical products.
In particular, this is by teaching methods and tools for developing user interfaces for cutting-edge user interface devices, by volunteering as a ``super user'' in our campus maker space, and by mentoring high school students in completing sustainable energy labs.

There are a number of other ways that I have been active in supporting design understanding and learning in the Berkeley community.
As one or the coordinators for the Berkeley Institute of Design seminar, where I help attract visiting speakers from institutions across the country to talk about a variety of design-related topics from fabrication to mixed initiative interfaces.
I have mentored on sustainability-related labs at Berkeley High School.
Finally, I regularly volunteer, where I help students learn to use lab equipment and keep the space open so they can pursue their projects.

(include that I also worked on writing the midterm and giving a lecture as Head TA)

\section{References}
\printbibliography[heading=none]

\end{document}
