\documentclass[12pt]{memoir}
\usepackage{common}

\addbibresource{references}

\title{}
\author{Andrew Head}

\begin{document}

\definition{Title}{Personal, Relevant Background and Future Goals Statement}

\definition{Author}{Andrew Head. PhD Student, UC Berkeley}

Better tools for end user developers can lead to better personal software and self-efficacy.
My research develops tools to help programmers make sense of and benefit from documentation.
% My past work has demonstrated my commitment to two of the NSF's criteria of Intellectual Merit and Broader Impacts.
My recent work shows \textbf{intellectual merit}---it has been recognized at a major conference on end-user programming with an Honorable Mention.
In research and beyond, I have pursued \textbf{broader impacts}---as head TA for a user interface design course and a volunteer in the campus hackerspace, I help others learn to build interactive artifacts with software.

\section{Personal Aims and Future Goals}

When I accepted admission to graduate school, I was interning as a software engineer with a Silicon Valley mobile advertising company.
It was clear that the skills I was acquiring were insufficient for me to reliably and methodically envision new technologies to satisfy human needs.
Academic study provided me with an opportunity to build and critically examine interfaces to empower people in technical tasks.

% I already feel closer to that goal, and this remains my core goal for the coming years of graduate school---developing tools and refining theory hand-in-hand to better support end user developer information seeking and programming.
By the conclusion of my PhD, I hope to have contributed to the study of end user developers in two ways.
First, I hope to contribute to theory that explains end user challenges and approaches to problem solving.
Second, I hope to build concrete end user development tools that enable new research and contribute value to a larger programmer community.

% One of the unquestioned motivations of HCI research throughout its history is that people working with the right digital tools can better author creative works for personal use.
% Digital tools that people have for building things for themselves can be improved, and can help people adapt the right design and implementation practices that will help them productively and effectively do what they need.
% Spreadsheets can enable accountants to perform complex, aggregate calculations (ref); tools for authoring CAD macros can reduce engineers' time replicating features for new models
% My research takes this belief as a starting point.

\if 0
My research and productive self has been working towards three major ends.
I have made progress towards each of these three ends.
My efforts in the next four years will continue to work towards bringing about success in each of these three ends.
\fi

\section{Intellectual Merit}

% This work is currently supported under an NSF CAREER grant on expertise sharing for end user programmers.
% The focus of my research and how this ties into the larger theme.
In my current work, I build new forms of context-relevant documentation to support end user developers.
People engage in end user programming when they program for personal rather than public use~\cite{ko_state_2011}.
While it has been estimated that there is 90 million end user programmers in the American workplace~\cite{scaffidi_estimating_2005}, it has also been shown that end user software engineering practices are haphazard compared to professional ones~\cite{ko_state_2011}.

Currently, I focus on building better documentation to support ``opportunistic''~\cite{clarke_what_2007} development of this group.
Typical reference documentation is often insufficient for end users trying to pick up new languages and APIs or approaching new types of tasks on a short schedule with a limited set of software engineering skills.
% A good deal of this inspirations comes not only from my own development experience as a professional programmer, but also from informal interviews with a handful of friends, and confirmation from literature for both professional programmers and end user programmers (ref) (ref).

This work has resulted in the Tutorons project.
The key insight of this work was that although programming tutorials may omit details about code snippets that programmers need to know, such information can be synthesized and packaged into a context-relevant form automatically.
In the Tutorons project, we built web servers that can detect code snippets for languages that often appear embedded in single lines online tutorials like command lines or regular expressions.
We elicited design and implementation patterns for developing context-relevant, on-demand programming documentation for found code snippets.
The work has been recently presented at VL/HCC and has received an Honorable Mention~\cite{head_tutorons_2015}.

In addition to my work on end user software engineering, I have published as a co-author in research on tangible user interfaces at CHI, a premiere HCI conference~\cite{savage_lamello_2015}.
I have also published as first author in serious games applied to an intelligent tutoring context~\cite{head_tonewars_2014}.

\section{Broader Impacts}

My research focuses on the NSF broader impact of \textbf{increasing public engagement with science and technology} by developing interfaces aimed to improve the efficacy of the many people who program as a non-professional practice.
Locally, I have worked to \textbf{improve STEM education} as an instructor for a user interface design course, volunteer in a campus hackerspace, and mentor at a local high school.

I served as the head TA for CS160, Berkeley's upper division user interface design course at Berkeley.
In this course, students learn to brainstorm, pitch, develop and present an interface developed using human-centered design.
Students are taught need finding, problem definition, interface programming for state of the art hardware, and presentation skills.

As head TA, I developed materials to support active learning for a class that emphasizes learning by doing.
I and a fellow TA wrote and led new lab-style learning activities and lectures to introduce programming skills such as wireless communication, databases, and location services for smartwatch technology.
At our sections, we planned activities for practicing presentations through wordplay and skits, and thinking through problems involving models of human performance like Fitts' Law.
We saw historically quite high attendance rates throughout the semester at the semi-weekly review sessions, with around 50\% attendance throughout the first two-thirds of the summer term.

Anecdotally, I have always been interested in building large systems used by many.
With a team of friends, I led programming efforts for \emph{King's Ascent}, a Flash game that gained over 70,000 plays on major game portals.
At past positions at Timesys Corporation and Quettra, Inc., I have both served on customer support, developer on professional services projects, and as an engineer deploying to dozens of machines collecting gigabytes of data.

In three other roles, I have worked to help foster a rich community of design research and practice at Berkeley.
As a ``Super User'' at Berkeley's Invention Lab, a student hackerspace, my main impact is keeping the lab open and teaching students to use the hardware for rapid prototyping so they can realize their ideas as physical devices.
I have volunteered at the Berkeley High School as a mentor in sustainability-related lab activities in a remedial-level classroom to engage with the larger community on pressing technological issues.
By co-organizing the Berkeley Institute of Design weekly seminar, I invite accomplished students and faculty in design research to speak, and arrange meetings between visiting scholars and our PhD students and faculty members.

\section{References}
\printbibliography[heading=none]

\end{document}
