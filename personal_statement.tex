\documentclass[12pt]{memoir}
\usepackage{common}

\addbibresource{references}

\title{}
\author{Andrew Head}

\begin{document}

\definition{Title}{Andrew Head: Personal Statement}

\definition{Author}{Andrew Head, PhD Student, UC Berkeley}

One of the beliefs underlying much modern HCI research is that people provided with the right digital tools can better serve themselves at lower cost than if somoene else were hired to do the task.
Digital tools that people have for building things for themselves can be improved, and can help people adapt the right design and implementation practices that will help them producively and effectively do what they need.
My research takes this belief as a starting point.

My research and productive self has been working towards three major ends.
I have made progress towards each of these three ends.
My efforts in the next four years will continue to work towards bringing about success in each of these three ends.

An introduction --- how I am supported now.
I am currently supported under an NSF CAREER grant on expertise sharing for end user programmers.

The focus of my research and how this ties into the larger theme.
I build tools for end user software engineering.
Currently I focus on building better documentation, which is heavily used by programmers of all backgrounds, but frankly insufficient for programmers trying to pick up new languages and APIs, or who are trying to approach new types of tasks.

With inspiration gained from Burnett \& Myers, I place an emphasis on grounding the research artifacts I build in established theory.
For example, the micro-explanations we used for Tutorons took inspiration from minimal instruction (Carroll 1990) and layered instruction (Farkas 1998) to build adaptive documentation that inserted additional detail into web tutorials on demand, while retaining only the information that was expected to be immediately relevant and helpful in error recognition and recovery.
This also draws on the principle of attention investment --- that users may be more likely to engage with and learn from documentation when there is a much smaller investment to be made in locating and understanding the relevant documentation.

In addition to my work on end user software engineering, I have also published as a co-author on fabrication research~\cite{savage_lamello_2015}.
Fabrication is related to this overarching goal of helping people serve themselves with powerful, expressive digital tools.
I have also published in a conference on intelligent tutoring systems on serious games applied to language learning~\cite{head_tonewars_2014}.

My past work experience has influenced both how I implement my work, and my metrics for success on my work.

During a past one-year internship, I gained familiarity with using the UNIX shell and supporting customers on low-level Linux compilation and commands such as kernel debugging and building custom Linux kernels.
From this, I gained the ability to work with computer systems on a fine-grained level and both understand and develop systems level software.

During another internship with Quettra, Inc., I developed infrastructure and tools for maintaining, monitoring, scaling and deploying software to thirty servers that were accepting daily traffic from over one million mobile phones.

While the success of my academic research can in many cases be evaluated on the success of its prototypes in demonstrating improved quality of software and ease of use among programmers, I'm interested in building shareable artifacts.
In its current form, this influenced my choice to build Tutorons as backend servers that could be flexibly implemented in any language with a loose architecture, and the implementation of Tutorons viewing as both a Javascript library that could be included on any arbitrary webpage or superimposed as a browser plugin.
All of the following is available for download through open source currently.
I am actively working with an undergraduate that I hope could become a maintainer for this project as we move forward.

Through my teaching and outreach experience, I have taken hands-on efforts to share an understanding of the work practices and technical tools necessary to build meaningful technical products.

I served as the Head TA for CS160, the upper division user interface design course at Berkeley.
While most courses undergraduates EECS majors take at Berkeley focus on the programming core of algorithms, machine learning, database implementation, and largely technical considerations, user interface design provides several unique challenges to students.
They must brainstorm, pitch, develop and present their own idea.
Through this process, they not only practice need finding, but also problem defintion, rapidly learn how to use new APIs for new hardware (in this case the Moto 360 Android smartwatch), and how to effectively convey their work at public presentations.

For this course, I developed new materials to support active learning in a class that emphasizes learning by doing.
I and my co-TA developed new lab-style learning activities with introductions to Android APIs for databases and location services, and tutorials on wireless connectivity between smartwatches and smartphones.
Our sections focused on giving chances for active practice of producing skits and wordplay for presentations, and working through problems related to models of human performance (Fitts' Law).
As a result, we saw historically quite high attendance rates throughout the semester, with around 50\% attendance through 2/3 of the summer term, huge improvements in the quality of student presentations, and some exceptional final projects from students (link to Valkyrie's site from here).

(include that I also worked on writing the midterm and giving a lecture as Head TA)

In two other roles, I have worked to help people understand new scientific knowledge or enable others to develop more.
These two roles also have two distinctly different flavors of students in terms of their backgrounds and technical proficiency, which highlights the importance of being able to develop conceptual knowledge equally consumable by students and practitioners of many different backgrounds.

As Inventioneer in the Invention Lab at Berkeley, the greatest impact of what I do is allowing others to work on their projects by being present in the lab during hours where no one else is looking after the lab.
On request, I help students brainstorm about electrical components they can use to sense or actuate for their projects, or troubleshoot the lab equipment (3D printers and vinyl cutters).
I regularly hold lab hours during evenings or weekends to enable those developing under time constraints to have more time making.

By co-coordinating the Berkeley Institute of Design lunch, I both introduce visiting professors and PhD students to the research at Berkeley, and also connect BiD members to these scholars.

\section{References}
\printbibliography[heading=none]

\end{document}
