\documentclass[12pt]{memoir}
\usepackage{common}

\addbibresource{references}

\title{}
\author{Andrew Head}

\begin{document}

\begin{center}
Personal, Relevant Background and Future Goals Statement
\end{center}
\vspace{-1ex}

\definition{Author}{Andrew Head. PhD Student, UC Berkeley}

% Introduction to the Pathos of My Work

\textbf{Code is awful.}
Code is wonderful.
It is an impediment to all progress.
It is an enabler of amazing feats of automation.
Code keeps us up at night when we grapple with the hopelessness of discovering that one line of code that causes unexpected behavior.
Code keeps us up at night when we furiously type out hundreds of rapid-fire lines for a prototype of a dream system.
Code is fantastic.
And abysmal.
And great.

There are 90 million estimated programmers in the country today when you include those who aren't professionals.
Each of us is driven by necessity, and sometimes passion or curiosity, to compute.
Though it only takes a quick visit to the nerd-centric web comics ``XKCD'' or ``PhD Comics'' to realize an inconvenient truth---
the code that we write as academics, students, and even computation practitioners is hilariously awful.

During these moments where we need a new program, we need all the help we can get to comprehend and adapt that code.
% We also need to recognize that programming is one of the most cognitively demanding things any one of us can do.
This is where I see my role in my academic and engineering communities.
% I place myself squarely within the camps of passion, necessity, and curiosity.
As a late-comer to programming (I discovered MATLAB when I was 19), code is what ended my days wandering as an academic nomad and led me into an intense preoccupation with how things work and how to build.
% Ever since, the satisfaction has never left from adding a new programming language to my portfolio, picking up a new machine learning algorithm, or launching a server with a quirky web app.
And after six years of working with glorious, awful code, I have gained a certainty that people must be encouraged and inspired to concentrate and struggle through their experiences with code.
It is only through focus in one of the most cognitively demanding tasks that we expand the bounds of our understanding and capability to design with the digital world.
% Over time, a desire to learn and build has evolved into a mission to let other people discover their own late-life love for code, if they choose, to find empowerment in the ability to understand and create digital systems.

% Research Evolution and Progress

% This inspires my recent, current, and future research to improve our understanding of programming in the wild and to understand the design of interventions that help people make better use of code outside the classroom.

Appropriately, my first academic work was on an end-user programming system---that is, an interface that sought to lower the cost of programming for those developing for personal use.
The system was called e-Chimera, which was an end-user programming environment that enabled researchers to create, deploy, and analyze in-the-wild mobile experiments.
It was only after I arrived at graduate school that I recognized the value of enabling large-scale in-situ mobile experiments, and how important it is to make sure that social science researchers have the ability to work with digital tools to gain new insights about people's behavior outside of the lab (a topic I am indeed interested in today with my desire to understand in-the-wild example usage).
What I contributed to this project was the design of interactions in the system's graphical editors and assisting in conducting a user study.

My research continued through two other sub-disciplines of HCI\@.
My first researcher as a first author was primary on ToneWars, an interdisciplinary work that focused on a rich game-based medium for language learning~\cite{head_tonewars_2014}.
It was during this research that I got my first experience interviewing experts and integrating novel pedagogy into a user interface.
This was my first published and presented academic work.
My contributions to the field were evaluating gameplay mechanisms to elicit how the app motivated native speakers and second-language learners of Chinese to collaborate in a language learning experience helpful to second-language learners.

I have also studied fabrication techniques for easing the design of tangible interfaces~\cite{savage_lamello_2015}.
This was during my initial collaborations at Berkeley where I was getting my footing as a researcher.
As we were preparing camera-ready revisions over winter break, I conducted some of what I felt was the densest period of experiment-driven work.
We were attempting to elicit the causes of vibrational noise in the body of 3D-printed input devices.
I collected dozens of recordings, and spent hours labeling peak frequencies and creating visualizations that would help us to elicit the cause of failed classification rates based on the acoustic signals the controls produced.
Our contributions to the field was a framework for designing passive acoustic tangible input devices that could be fabricated on a 3D printer.

Today, my research finds itself squarely in the advocacy of programmers once again~\cite{head_tutorons_2015}.
My current work is on Tutorons.
My proposal aims to gain more inspiration on what is wrong in modern programming practice and developing some of the missing tools to better support the cognitive shortcuts that programmers try to take when programming.
This current work has shown the ability to automatically explain online code found in the browser.
It has been a combination of natural language generation and data-driven understanding of common for a large-scale corpus of online code examples.
After a recent presentation at VL/HCC, a nomination for Best Paper and excited discussions with my colleagues, I am convinced that the problems of programming information seeking are far from solved.
My contributions to the field were an exploration of the design and implementation space of automatically detecting and explaining code found online through a variety of methods, including automatic explanations and collections of usage examples.

% Outreach: Teaching experiences

I served as the head TA for CS160, Berkeley's upper division user interface design course at Berkeley.
In this course, students learn to brainstorm, pitch, develop and present an interface developed using human-centered design.
Students are taught need finding, problem definition, interface programming for state of the art hardware, and presentation skills.

As head TA, I developed materials to support active learning for a class that emphasizes learning by doing.
I and a fellow TA wrote and led new lab-style learning activities and lectures to introduce programming skills such as wireless communication, databases, and location services for smartwatch technology.
At our sections, we planned activities for practicing presentations through wordplay and skits, and thinking through problems involving models of human performance like Fitts' Law.
We saw historically quite high attendance rates throughout the semester at the semi-weekly review sessions, with around 50\% attendance throughout the first two-thirds of the summer term.

As a CS160 GSI, I saw several successes in helping students overcome barriers with code.
After the final poster session, one of my students told me that I was the best TA he has ever had.
One team that I worked closely with through the ideation and prototyping process won first prize at our poster session.
After a whimsical session I led where I instructed students on how to incorporate playfulness into their presentations, I in fact saw skits and wordplay make their way into student presentations.
I also coordinated students to collaborate to lead each other to solutions to each others' problems during office hours for a challenging programming project.
It was during this time that I developed much of my motivation to help programmers critically engage with code that they are reusing.

% Outreach: Working

As a worker, I myself have encountered, overcome, and encountered again the problems of working with code.
Working with a new language or framework can be exceedingly difficult, a lesson I learned at Quettra.
From working with Linux, it's clear that strong knowledge of theory behind a set of tools and an attentiveness to detail are very important.
When working on the Astute project, I noticed many of the same problems from my teammates who were working with code.

Anecdotally, I have always been interested in building large systems used by many.
With a team of friends, I led programming efforts for \emph{King's Ascent}, a Flash game that gained over 70,000 plays on major game portals.
At past positions at Timesys Corporation and Quettra, Inc., I have both served on customer support, developer on professional services projects, and as an engineer deploying to dozens of machines collecting gigabytes of data.
% Outreach: Community Involvement

% I will be frank and up-front in specifying that in no typical way do I represent an underrepresented group.
As a Caucasian mid-20s male programmer in Silicon Valley, I am indeed emblematic of the majority.
Where I lack in leading from within an underrepresented group, I have taken upon myself a commitment to foster a supportive, productive culture of design, craft, and building at UC Berkeley.

Since January of this year, I have worked with Cesar Torres to coordinate the Berkeley Institute of Design's weekly seminar, inviting speakers from diverse disciplines, offering them tours and, most importantly, connecting them to the students and faculty that would most benefit from a one-on-one with the speaker during their visit.
I led the organization of HCI events for last year's Visit Day for incoming students, organizing our outdoor activities and corralling the student and faculty for the group.
I organized a reception for the UIST PC committee during their visit to Berkeley, involving students from over fifteen student projects at a mixer with some of the top research talent in our field.
I volunteer at the CITRIS Invention Lab, an on-campus hackerspace, to keep the lab open for students to prototype their designs on the fabrication hardware and by assembling electronic components.

During my time at the University of Pittsburgh, I often copy-edited demo video scripts and paper submissions for the peers in my group, the majority of whom were ESL speakers.
I have also volunteered in the past with the Berkeley High School to mentor on sustainability-centered labs.
The results of each of these was.

% Longer-Term Ambitions

\emph{%
Working with code is not enough to achieve my goals to encourage and enable programmers to concentrate on and struggle through their experiences with code, particularly in the wild.
Only in academic research is learning, cognitive ability, and problem-solving a topic of so much interest that I can legitimately study this and advocate for programmers' needs while getting paid for it.
}

When I accepted admission to graduate school, I was interning as a software engineer with a Silicon Valley mobile advertising company.
It was clear that the skills I was acquiring were insufficient for me to reliably and methodically envision new technologies to satisfy human needs.
Academic study provided me with an opportunity to build and critically examine interfaces to empower people in technical tasks.

% I already feel closer to that goal, and this remains my core goal for the coming years of graduate school---developing tools and refining theory hand-in-hand to better support end user developer information seeking and programming.
By the conclusion of my PhD, I hope to have contributed to the study of end user developers in two ways.
First, I hope to contribute to theory that explains end user challenges and approaches to problem solving.
Second, I hope to build concrete end user development tools that enable new research and contribute value to a larger programmer community.

\if 0
Better tools for end user developers can lead to better personal software and self-efficacy.
My research develops tools to help programmers make sense of and benefit from documentation.
% My past work has demonstrated my commitment to two of the NSF's criteria of Intellectual Merit and Broader Impacts.
My recent work shows \textbf{intellectual merit}---it has been recognized at a major conference on end-user programming with an Honorable Mention.
In research and beyond, I have pursued \textbf{broader impacts}---as head TA for a user interface design course and a volunteer in the campus hackerspace, I help others learn to build interactive artifacts with software.
\fi

\section{Personal Aims and Future Goals}

% One of the unquestioned motivations of HCI research throughout its history is that people working with the right digital tools can better author creative works for personal use.
% Digital tools that people have for building things for themselves can be improved, and can help people adapt the right design and implementation practices that will help them productively and effectively do what they need.
% Spreadsheets can enable accountants to perform complex, aggregate calculations (ref); tools for authoring CAD macros can reduce engineers' time replicating features for new models
% My research takes this belief as a starting point.

\section{References}
\printbibliography[heading=none]

\end{document}
